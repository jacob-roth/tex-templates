%
%%
%%% jmr commands
%%
%

\NeedsTeXFormat{LaTeX2e}
\ProvidesPackage{jmrcommands}[2022/02 Commands etc]

\RequirePackage{imakeidx}
\RequirePackage{xstring}
\RequirePackage{xcolor}
\newcommand{\mathopfont}{\mathrm}
\DeclareOption{mathopsf}{\renewcommand{\mathopfont}{\mathsf}}
\DeclareOption{mathoprm}{\renewcommand{\mathopfont}{\mathrm}}
\DeclareOption*{\PackageWarning{jmrcommands}{Unknown \CurrentOption}}
\ProcessOptions\relax

%
%%
%%% commands helper
%%
% 

%%% calcing expression (see \seq)
%---
\ExplSyntaxOn
  \cs_new_eq:NN \CALC \fp_eval:n
\ExplSyntaxOff

\newcounter{argcnt}
\makeatletter
\newcommand{\newarray}[2]{% \newarray{<array>}{<csv list>}
  \setcounter{argcnt}{0}% Restart argument count
  \renewcommand{\do}[1]{% With each element do...
    \stepcounter{argcnt}% Next arg
    \expandafter\gdef\csname #1@\theargcnt\endcsname{##1}% Store arg in \<array>@<num>
  }%
  \docsvlist{#2}% Store array
  \expandafter\xdef\csname #1@0\endcsname{\theargcnt}% Store array element count
}
\newcommand{\getelement}[2]{% \getelement{<array>}{<num>}
  \@ifundefined{#1@0}% <array> not defined
    {\@latex@error{Array #1 undefined}\@ehc}{}
  \expandafter\ifnum\csname #1@0\endcsname<#2 % <num> out of range
    \@latex@error{Array #1 only has \csname #1@0\endcsname\space elements}\@ehc
  \fi
  \csname #1@#2\endcsname% Print element
}
\newcommand{\calc}[1]{\expandafter\CALC\expandafter{#1}}% calculations
\makeatother
%---

%%% footnote in tcolorbox
%---
\makeatletter
\let\original@footnote\footnote
\newcommand{\align@footnote}[1]{%
  \ifmeasuring@
    \chardef\@tempfn=\value{footnote}%
    \footnotemark
    \setcounter{footnote}{\@tempfn}%
  \else
    \iffirstchoice@
      \original@footnote{#1}%
    \fi
  \fi}
\pretocmd{\start@align}{\let\footnote\align@footnote}{}{}
\makeatother
%---

%
%%
%%% custom commands
%%
%

%%% math operators
\DeclareMathOperator*{\argmin}{arg\,min}
\DeclareMathOperator*{\Argmin}{Arg\,min}
\DeclareMathOperator*{\find}{\mathopfont{find}}
\DeclareMathOperator*{\argmax}{arg\,max}
\DeclareMathOperator*{\Argmax}{Arg\,max}
\DeclareMathOperator*{\arginf}{arg\,inf}
\DeclareMathOperator*{\argsup}{arg\,sup}

%%% symbols
\DeclareFontFamily{OMX}{MnSymbolE}{}
\DeclareFontShape{OMX}{MnSymbolE}{m}{n}{
    <-6>  MnSymbolE5
   <6-7>  MnSymbolE6
   <7-8>  MnSymbolE7
   <8-9>  MnSymbolE8
   <9-10> MnSymbolE9
  <10-12> MnSymbolE10
  <12->   MnSymbolE12}{}
\DeclareSymbolFont{mnlargesymbols}{OMX}{MnSymbolE}{m}{n}
\SetSymbolFont{mnlargesymbols}{bold}{OMX}{MnSymbolE}{b}{n}
\DeclareMathDelimiter{\llangle}{\mathopen}{mnlargesymbols}{'164}{mnlargesymbols}{'164}
\DeclareMathDelimiter{\Rangle}{\mathclose}{mnlargesymbols}{'171}{mnlargesymbols}{'171}

%
%%
%%% custom shortcuts
%%
%

%%% general math
\newcommand{\R}{\mathbb{R}}
\newcommand{\C}{\mathbb{C}}
\newcommand{\N}{\mathbb{N}}
\newcommand{\Z}{\mathbb{Z}}
\newcommand{\Q}{\mathbb{Q}}
\newcommand{\Symmetric}{\mathbb{S}}
\newcommand{\real}[1]{\mathfrak{Re}\!\left(#1\right)}
\newcommand{\imag}[1]{\mathfrak{Im}\!\left(#1\right)}
\newcommand{\ind}{\mathds{1}}
\newcommand{\dee}{\mathrm{d}}
\newcommand{\dd}[1]{\frac{\dee}{\dee #1}}
\newcommand{\dom}{\mathopfont{dom}}
\DeclarePairedDelimiterXPP\relint[1]{\operatorname{\mathopfont{relint}}}[]{}{#1}% https://tex.stackexchange.com/questions/635277/newcommand-with-option-for-bracket-size
\DeclarePairedDelimiterXPP\int[1]{\operatorname{\mathopfont{int}}}[]{}{#1}% https://tex.stackexchange.com/questions/635277/newcommand-with-option-for-bracket-size
\DeclarePairedDelimiter\abs{\lvert}{\rvert}%optional param size, e.g. \abs[big]{x^2}
\DeclarePairedDelimiterXPP\cl[1]{\operatorname{\mathopfont{cl}}}[]{}{#1}% https://tex.stackexchange.com/questions/635277/newcommand-with-option-for-bracket-size
\DeclarePairedDelimiterXPP\dom[1]{\operatorname{\mathopfont{dom}}}[]{}{#1}% https://tex.stackexchange.com/questions/635277/newcommand-with-option-for-bracket-size
\DeclarePairedDelimiter\ceil{\lceil}{\rceil}
\DeclarePairedDelimiter\floor{\lfloor}{\rfloor}
\DeclarePairedDelimiterXPP\sign[1]{\operatorname{\mathopfont{sign}}}[]{}{#1}% https://tex.stackexchange.com/questions/635277/newcommand-with-option-for-bracket-size
\newcommand{\seq}[3]{#1_{#2},#1_{\calc{#2+1}},\ldots,#1_{#3}} % \seq{X}{1}{n} = X_1,X_2,...X_n
\newcommand{\lbar}[1]{\underbar{$#1$}}
\DeclarePairedDelimiterXPP\vol[1]{\operatorname{\mathopfont{vol}}}[]{}{#1}% https://tex.stackexchange.com/questions/635277/newcommand-with-option-for-bracket-size
\newcommand{\Lip}[1]{\mathopfont{Lip}\left(#1\right)}
\newcommand{\contradict}{\Rightarrow\!\Leftarrow}
\newcommand{\Id}{\mathopfont{Id}}
\newcommand{\extended}[1]{\overline{\underline{#1}}}

%%% optimization
\DeclarePairedDelimiter{\LR}{[}{]}
\DeclareDocumentCommand \prox { o m m } {% optional mandatory mandatory: https://tex.stackexchange.com/questions/1742/automatic-left-and-right-commands
  \IfNoValueTF {#1} {%
    \textbf{prox}_{#2}\LR{#3}%
  }{%
    \textbf{prox}_{#2}\LR[#1]{#3}%
  }%
}
\DeclareDocumentCommand \proj { o m m } {% optional mandatory mandatory: https://tex.stackexchange.com/questions/1742/automatic-left-and-right-commands
  \IfNoValueTF {#1} {%
    \textbf{proj}_{#2}\LR{#3}%
  }{%
    \textbf{proj}_{#2}\LR[#1]{#3}%
  }%
}
% \newcommand{\norm}[1]{\left\lVert#1\right\rVert}
\DeclarePairedDelimiter\norm{\lVert}{\rVert}%optional param size, e.g. \norm[big]{x^2}: https://tex.stackexchange.com/questions/297253/norm-symbol-have-different-sizes-in-equation
% \newcommand{\inner}[2]{\left\langle#1{\,,\,}#2\right\rangle}
\DeclarePairedDelimiterX{\inner}[2]{\langle}{\rangle}{#1, #2}
\DeclarePairedDelimiterXPP\len[1]{\operatorname{\mathopfont{len}}}[]{}{#1}% https://tex.stackexchange.com/questions/635277/newcommand-with-option-for-bracket-size
\DeclarePairedDelimiterXPP\conv[1]{\operatorname{\mathopfont{conv}}}[]{}{#1}% https://tex.stackexchange.com/questions/635277/newcommand-with-option-for-bracket-size
\DeclarePairedDelimiterXPP\convexhull[1]{\operatorname{\mathopfont{cvx}}}[]{}{#1}% https://tex.stackexchange.com/questions/635277/newcommand-with-option-for-bracket-size
\DeclarePairedDelimiterXPP\concavehull[1]{\operatorname{\mathopfont{ccv}}}[]{}{#1}% https://tex.stackexchange.com/questions/635277/newcommand-with-option-for-bracket-size
\newcommand{\primal}{(\mathopfont{P})}
\newcommand{\dual}{(\mathopfont{D})}
\newcommand{\feas}{\mathopfont{F}}
\newcommand{\opt}{\mathopfont{OPT}}
\newcommand{\alg}{\mathopfont{Alg}}
\newcommand{\rlax}{(\mathopfont{R})}
\newcommand{\primaldual}{(\mathopfont{P}\text{-}\mathopfont{D})}
\newcommand{\existence}{(\mathopfont{E})}
\newcommand{\poly}[1]{\mathfrak{P}\!\left[#1\right]}
\newcommand{\tangentcone}[2]{\mathcal{T}\!\left(#1;#2\right)}
\newcommand{\linearcone}[2]{\mathcal{L}\!\left(#1;#2\right)}
\newcommand{\feasiblecone}[2]{\mathcal{F}\!\left(#1;#2\right)}
\DeclarePairedDelimiterXPP\epi[1]{\operatorname{\mathopfont{epi}}}[]{}{#1}% https://tex.stackexchange.com/questions/635277/newcommand-with-option-for-bracket-size
\DeclarePairedDelimiterXPP\cone[1]{\operatorname{\cone{epi}}}[]{}{#1}% https://tex.stackexchange.com/questions/635277/newcommand-with-option-for-bracket-size
\newcommand{\vi}{\mathopfont{VI}}
\newcommand{\xhat}{\hat{x}}
\newcommand{\xbar}{\bar{x}}
\newcommand{\xstar}{x^\star}
\newcommand{\xstarr}{x^*}

%%% linear algebra
\DeclarePairedDelimiterXPP\range[1]{\operatorname{\mathopfont{range}}}[]{}{#1}% https://tex.stackexchange.com/questions/635277/newcommand-with-option-for-bracket-size
\DeclarePairedDelimiterXPP\nullspace[1]{\operatorname{\mathopfont{null}}}[]{}{#1}% https://tex.stackexchange.com/questions/635277/newcommand-with-option-for-bracket-size
\DeclarePairedDelimiterXPP\trace[1]{\operatorname{\mathopfont{trace}}}[]{}{#1}% https://tex.stackexchange.com/questions/635277/newcommand-with-option-for-bracket-size
\DeclarePairedDelimiterXPP\det[1]{\operatorname{\mathopfont{det}}}[]{}{#1}% https://tex.stackexchange.com/questions/635277/newcommand-with-option-for-bracket-size
\DeclarePairedDelimiterXPP\rank[1]{\operatorname{\mathopfont{rank}}}[]{}{#1}% https://tex.stackexchange.com/questions/635277/newcommand-with-option-for-bracket-size
\DeclarePairedDelimiterXPP\diag[1]{\operatorname{\mathopfont{diag}}}[]{}{#1}% https://tex.stackexchange.com/questions/635277/newcommand-with-option-for-bracket-size
\DeclarePairedDelimiterXPP\vec[1]{\operatorname{\mathopfont{vec}}}[]{}{#1}% https://tex.stackexchange.com/questions/635277/newcommand-with-option-for-bracket-size

%%% probability
%--- prob argument: https://tex.stackexchange.com/a/636626/144086
\DeclarePairedDelimiterX{\probargument}[1]{[}{]}{\probargumentA{#1}}
\NewDocumentCommand{\probargumentA}{>{\SplitArgument{1}{|}}m}{%
  \probargumentB#1%
}
\NewDocumentCommand{\probargumentB}{mm}{%
  \IfNoValueTF{#2}{% not conditional
    #1%
  }{% conditional
    #1\;\delimsize|\;#2%
  }%
}
%--- 
\NewDocumentCommand{\prob}{e{_}}{% prob
  \mathopfont{P}\IfValueT{#1}{_{#1}}\probargument
}
\NewDocumentCommand{\prob}{e{_}}{% expect
  \mathopfont{E}\IfValueT{#1}{_{#1}}\probargument
}
\NewDocumentCommand{\variance}{e{_}}{% variance
  \mathopfont{Var}\IfValueT{#1}{_{#1}}\probargument
}
\NewDocumentCommand{\covariance}{e{_}}{% covariance
  \mathopfont{Cov}\IfValueT{#1}{_{#1}}\probargument
}
\newcommand{\field}{\mathcal{F}}
\newcommand{\borel}[1]{\mathcal{B}({#1})}
\newcommand{\algebra}{\mathcal{A}}
\newcommand{\dtmc}{(X_n)_{n\geq0}}
\newcommand{\ctmc}{(X(t))_{t\geq0}}
\newcommand{\iid}{\overset{\mathopfont{iid}}{\sim}}
\newcommand{\dist}[1]{\sim\mathopfont{#1}}% https://www.overleaf.com/project/5eacf06d2a17600001697726
\newcommand{\pr}{\mathopfont{P}}
\newcommand{\law}{\mathopfont{Law}}
% convergence
\DeclareMathOperator*{\pto}{\overset{p}{\to}}
\DeclareMathOperator*{\dto}{\overset{d}{\to}}
\DeclareMathOperator*{\asto}{\overset{a.s.}{\to}}
\DeclareMathOperator*{\pequal}{\overset{p}{=}}
\DeclareMathOperator*{\dequal}{\overset{d}{=}}
\DeclareMathOperator*{\asequals}{\,\overset{a.s.}{=}\,}
\DeclareMathOperator*{\plim}{plim}% https://tex.stackexchange.com/questions/125750/how-to-write-a-modified-limit-operator
% distributions
\DeclareMathOperator{\Multinom}{Multinom}
\DeclareMathOperator{\Bern}{Bern}
\DeclareMathOperator{\Unif}{Unif}
\DeclareMathOperator{\Geom}{Geom}
\DeclareMathOperator{\Normal}{Normal}
\DeclareMathOperator{\Poi}{Poi}
\DeclareMathOperator{\Exp}{Exp}
\DeclareMathOperator{\Gam}{Gam}
\DeclareMathOperator{\Bin}{Binom}

%%% text
\newcommand{\highlight}[1]{{\color{teal}{#1}}}
\newcommand{\jr}[1]{\highlight{(#1)}}
\newcommand{\red}[1]{{\color{red}{#1}}}
\newcommand{\blue}[1]{{\color{blue}{#1}}}
\newcommand{\purple}[1]{{\color{purple}{#1}}}
\newcommand{\orange}[1]{{\color{orange}{#1}}}
\newcommand{\teal}[1]{{\color{teal}{#1}}}
\definecolor{ForestGreen}{RGB}{34,139,34}
\newcommand{\green}[1]{{\color{ForestGreen}{#1}}}
\newcommand{\grey}[1]{{\color{gray}{#1}}}
\newcommand{\gray}[1]{{\color{gray}{#1}}}
\newcommand{\black}[1]{{\color{black}{#1}}}